\documentclass[letterpaper,10pt]{article}
\usepackage{amssymb}

\setlength{\oddsidemargin}{-0.3in}
\setlength{\textwidth}{6.8in}
%\setlength{\evensidemargin}{.15in}
\setlength{\topmargin}{-0.65in} \setlength{\textheight}{9.2in}

\pagestyle{empty}

\begin{document}

\begin{center}
{\bf {\LARGE{COMBINATORIA ALGEBRAICA}}}\\
\
\\ {\bf{\Large Tarea 3}}\\ \ \\
{\bf{\large{Federico Ardila}}}\\ \ \\ {\large Fecha de entrega: 27 de Febrero de 2003 }\\[.25in]
\end{center}


\large

\noindent {\sc Instrucciones: }{Entregue $4$ de los siguientes
problemas. El problema $5$ cuenta como $3$ problemas diferentes.
Cada problema vale entre $5$ y $10$ puntos, dependiendo de la
soluci\'{o}n y la dificultad del problema.}

\smallskip

%\smallskip
\begin{center}
\noindent {\sc Problemas }
\end{center}

%----------------------------comienzo enunciado

\begin{enumerate}


\item (EC, Ejercicio Suplementario 3.1) Sea $P$ un poset de $n$
elementos. Para cada $x \in P$, sea $c_x$ el n\'{u}mero de
elementos de $P$ que son menores o iguales que $x$. Sea $e(P)$ el
n\'{u}mero de extensiones lineales de $P$. Demostrar que
$$
e(P) \geq \frac{n!}{\prod_{x \in P} c_x}.
$$

\item Demuestre que el poset $\Pi_n$ de particiones de $[n]$ es un
l\'{a}tice. >Para qu\'{e} valores de $n$ es distributivo?

\item Si $\lambda = (\lambda_1, \lambda_2, \ldots)$ y $\mu =
(\mu_1, \mu_2, \ldots)$ son dos particiones de $n$, se dice que
$\lambda$ {\bf domina} a $\mu$ si $\lambda_1 + \cdots + \lambda_i
\geq \mu_1 + \cdots + \mu_i$ para todo entero positivo $i$.

Sea Par($n$) el poset de particiones de $n$, donde $\lambda \geq
\mu$ si y s\'{o}lo si $\lambda$ domina a $\mu$.

Demuestre que Par($n$) es un l\' {a}tice.

\item Sea $G$ un grafo finito cuyo conjunto de v\'{e}rtices es
$V$. Un subconjunto $U \subseteq V$ es {\bf $G$-conexo} si la
restricci\'{o}n de $G$ a $U$ es conexa.

Considere las particiones $\pi$ del conjunto $V$ tales que cada
parte de $\pi$ es $G$-conexa. Sea $L_G$ el poset de tales
particiones, ordenadas por refinamiento.

Demuestre que $L_G$ es un l\'{a}tice.

\item Sea $A$ el conjunto de palabras finitas cuyos d\'{\i}gitos
son $0$'s y $1$'s. Se define un orden parcial en $A$ de la
siguiente manera: las palabras que cubren a $a$ son las palabras
que se obtienen al cambiar un $1$ de $a$ por un $0$, y la palabra
que se obtiene al a\~{n}adir un $1$ al final de $a$.

\begin{enumerate}
\item Demuestre que $A$ es un poset graduado, y que el n\'{u}mero
de elementos en el nivel $i$ es igual a $F_i$, el $i$-\'{e}simo
n\'{u}mero de Fibonacci.

\item Demuestre que $A$ es un l\'{a}tice.

\item Demuestre que cualquier intervalo de $A$ es un l\'{a}tice
distributivo.
\end{enumerate}

\item (EC, Ejercicio 3.22b) Sea $L$ un l\'{a}tice distributivo
finito y $f:\Bbb N \rightarrow \Bbb N$ una funci\'{o}n tales que,
para cada $x \in L$, si $x$ cubre a exactamente $i$ elementos de
$L$, entonces a $x$ lo cubren exactamente $f(i)$ elementos de $L$.
Demuestre que $L \cong B_k$ para alg\'{u}n entero positivo $k$.


\end{enumerate}

%----------------------------fin enunciado



\end{document}
