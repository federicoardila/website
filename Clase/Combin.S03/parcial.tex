\documentclass[letterpaper,10pt]{article}
%\usepackage{amssymb}

\setlength{\oddsidemargin}{-0.3in}
\setlength{\textwidth}{6.8in}
%\setlength{\evensidemargin}{.15in}
\setlength{\topmargin}{-0.65in} \setlength{\textheight}{9.2in}

\pagestyle{empty}

\begin{document}

\begin{center}
{\bf {\Large{Parcial de Combinatoria Algebraica}}}\\
{\large Marzo 4,
2003 }\\[.25in]
\end{center}


\large

%\noindent {\sc Instrucciones: }{Los problemas requieren
%demostraci\'{o}n o justificaci\'{o}n completa. Respuestas sin
%justificaci\'{o}n no obtendr\'{a}n un puntaje superior al 10\%. No
%se permite el uso de calculadoras, textos o apuntes. }

\smallskip

%\noindent {\sc Tiempo M\'aximo: 2 Horas. }

%\smallskip
\begin{center}
\noindent {\sc Problemas }
\end{center}

%----------------------------comienzo enunciado

\begin{enumerate}

\item Decimos que una permutaci{\'o}n $a_1 \ldots a_n$ es {\bf
alternante} si $a_1 < a_2 > a_3 < a_4 > \cdots$. Sea $E_n$ el
n{\'u}mero de permutaciones alternantes de $[n]$.

\begin{enumerate}
\item Demuestre que
$$
2E_{n+1} = \sum_{k=0}^n {n \choose k} E_k E_{n-k}.
$$

\item Concluya que
$$
\sum_{n \geq 0} E_n \frac{x^n}{n!} = \tan x + \sec x.
$$

\item Use el resultado de la parte (b) para dar una
demostraci{\'o}n combinatoria de la identidad
$$
1 + \tan^2 x = \sec^2 x.
$$

\end{enumerate}

\item Sean $n$ y $k$ enteros positivos. Sea $a_k(n)$ el n{\'u}mero
de particiones de $n$ en partes que no son m{\'u}ltiplos de $k$.
Sea $b_k(n)$ el n{\'u}mero de particiones de $n$ donde ninguna
parte aparece $k$ o m{\'a}s veces.

Demuestre que $a_k(n) = b_k(n)$.

\item Sea $P$ un poset finito y sea $k$ el tama{\~n}o m{\'a}ximo
de una anticadena de $P$. Sean $A$ y $B$ dos anticadenas de $P$ de
$k$ elementos. Sean $C$ el conjunto de elementos maximales y $D$
el conjunto de elementos minimales del poset $A \cup B$ (con el
orden inducido por $P$). Demuestre que $C$ y $D$ son anticadenas
de $P$ de $k$ elementos.

\item Encuentre el n{\'u}mero de posets $P$ de $n$ elementos tales
que, para cada entero positivo $i$ con $1 \leq i \leq n-1$, $P$
tiene exactamente dos ideales de $i$ elementos. Dibuje los posets
$P$ y $J(P)$ que se obtienen en el caso $n = 4$.

\item

Sea $P$ un poset de $n$ elementos numerado naturalmente. Si $f_k$
es el n{\'u}mero de cadenas de $J(P)$ de $k+1$ elementos, el
$f$-polinomio de $J(P)$ es:
$$
f_{J(P)}(x) = \sum_k f_kx^k.
$$
Podemos ver las extensiones lineales de $P$ como permutaciones de
$[n]$. Si $w_k$ es el n{\'u}mero de extensiones lineales de $P$
con $k$ descensos, el $W$-polinomio de $P$ es:
$$
W_P(x) = \sum_k w_kx^k.
$$
Demostrar que
$$
W_P\left(\frac{x}{x+1}\right) = \frac{f_{J(P)}(x)}{(x+1)^n}.
$$
>Qu{\'e} dice esta afirmaci{\'o}n cuando $x \rightarrow \infty$?



{\bf Nota}. Una numeraci{\'o}n natural de $P$ es una
numeraci{\'o}n de los elementos de $P$ con los n{\'u}meros $1,
\ldots, n$ tal que, si $i < j$ en el poset, entonces el n{\'u}mero
asignado al elemento $i$ es menor que el n{\'u}mero asignado a
$j$.



\end{enumerate}


%----------------------------fin enunciado



\end{document}
