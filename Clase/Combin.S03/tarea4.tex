\documentclass[letterpaper,10pt]{article}
\usepackage{amssymb}

\setlength{\oddsidemargin}{-0.3in}
\setlength{\textwidth}{6.8in}
%\setlength{\evensidemargin}{.15in}
\setlength{\topmargin}{-0.65in} \setlength{\textheight}{9.2in}

\pagestyle{empty}

\begin{document}

\begin{center}
{\bf {\LARGE{COMBINATORIA ALGEBRAICA}}}\\
\
\\ {\bf{\Large Tarea 4}}\\ \ \\
{\bf{\large{Federico Ardila}}}\\ \ \\ {\large Fecha de entrega: 10 de Abril de 2003 }\\[.25in]
\end{center}


\large

\noindent {\sc Instrucciones: }{Entregue {\bf tres} de los
siguientes problemas. Adem{\'a}s, describa brevemente (en un
p{\'a}rrafo) el objetivo de su proyecto final. Cada problema vale
entre $5$ y $10$ puntos, dependiendo de la soluci\'{o}n y la
dificultad del problema.}

\smallskip

%\smallskip
\begin{center}
\noindent {\sc Problemas }
\end{center}

%----------------------------comienzo enunciado

\begin{enumerate}

\item Encuentre $\mu(\hat{0}, \hat{1})$ en el l{\'a}tice de
particiones $\Pi_n$.

(Pista: Recuerde que $\Pi_n$ es el l{\'a}tice de contracciones del
grafo completo $K_n$.)

\item EC, Ejercicio Suplementario 2.1(a).

\item Si $G$ es un grafo y ${\cal A}_G$ es su arreglo de
hiperplanos asociado, demuestre que $L_G \cong L_{{\cal A}_G}$.

\item Calcule el polinomio caracter{\'\i}stico del arreglo de $k$
hiperplanos en posici{\'o}n general en $\Bbb R^n$.

\item Encuentre una biyecci{\'o}n entre las funciones de parqueo
de longitud $n$ y los {\'a}rboles numerados de $n+1$ v{\'e}rtices.
(Ser{\'i}a excelente, pero no necesario, que su biyecci{\'o}n
asigne a cada funci{\'o}n de parqueo $(a_1, \ldots, a_n)$ con $a_1
+ \cdots + a_n = k$ un {\'a}rbol que tiene${n+1 \choose 2} - k$
inversiones.)


\end{enumerate}

%----------------------------fin enunciado



\end{document}
