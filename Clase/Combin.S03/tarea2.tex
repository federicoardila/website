\documentclass[letterpaper,10pt]{article}
\usepackage{amssymb}

\setlength{\oddsidemargin}{-0.3in}
\setlength{\textwidth}{6.8in}
%\setlength{\evensidemargin}{.15in}
\setlength{\topmargin}{-0.65in} \setlength{\textheight}{9.2in}

\pagestyle{empty}

\begin{document}

\begin{center}
{\bf {\LARGE{COMBINATORIA ALGEBRAICA}}}\\
\
\\ {\bf{\Large Tarea 2}}\\ \ \\
{\bf{\large{Federico Ardila}}}\\ \ \\ {\large Fecha de entrega: 11 de Febrero de 2003 }\\[.25in]
\end{center}


\large

\noindent {\sc Instrucciones: }{Entregue cinco de los siguientes
problemas. Cada problema vale entre $5$ y $10$ puntos, dependiendo
de la soluci\'{o}n y la dificultad del problema.}

\smallskip

%\smallskip
\begin{center}
\noindent {\sc Problemas }
\end{center}

%----------------------------comienzo enunciado

\begin{enumerate}

\item EC, Ejercicio 1.7

\item EC, Ejercicio 1.9

\item EC, Ejercicio 1.10

\item EC, Ejercicio Suplementario 1.1

\item EC, Ejercicio Suplementario 1.2

\item EC, Ejercicio Suplementario 1.8

\item EC, Ejercicio Suplementario 1.13

\item EC, Ejercicio Suplementario 1.24

\item Complete los detalles de EC, Proposici\'{o}n 1.3.19.

\item Demuestre que, para cada entero positivo $n$,
$$
\sum_{\pi \in S_n} q^{{\mathrm MAJ}(\pi)} = (1+q)(1+q+q^2) \cdots
(1+q+\cdots+q^{n-1}).
$$


\end{enumerate}

%----------------------------fin enunciado



\end{document}
